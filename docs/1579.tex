\subsection{题面}

%%%%%%%%%%%%%%%%%%%%%%%%%%%%%%%%%%%%%%%%%%%%%%%%%%%%%%%%%%%%%%%%%%%%%%%%%%%%%%%
\subsubsection{Time Limit}
2s

%%%%%%%%%%%%%%%%%%%%%%%%%%%%%%%%%%%%%%%%%%%%%%%%%%%%%%%%%%%%%%%%%%%%%%%%%%%%%%%
\subsubsection{Memory Limit}
128M

%%%%%%%%%%%%%%%%%%%%%%%%%%%%%%%%%%%%%%%%%%%%%%%%%%%%%%%%%%%%%%%%%%%%%%%%%%%%%%%
\subsubsection{题目}
有一天老彼得对小彼得说:“小彼得,我这里有 $t$ 个数组,每一个数组  $A_j$($1 \le
j \le t$),不妨设数组 $A_j$ 的长度为 $L_j$,数组的上界为 $R_j$(上界不一定在数
组里,但是大于等于所有的数组元素),那么我可以断言它的每一个元素 $a_i$($1 \le i
\le L_j$)满足 $2 \le a_i \le R_j$。”

老彼得知道小彼得最近学习了有手就行的算数基本定理(即:任何整数都存在,且只存在一
种的分解为质数乘积的形式),老彼得的妻子安娜·卡特琳娜因为一个让人羞愤的理由离开
了老彼得,老彼得突然对儿子的教育燃起了新的热情,他补充到:“我知道,给定一个数 $E
= p_1^{b_1}\cdot p_2^{b_2}\cdot \cdots \cdot p_n^{b_n}$,它有且只有这么一种表达
形式,所以 $S(E) = b_1 + b_2 + \cdots + b_n$ 的值也是唯一的,比如说 $72 = 2^3
\times 3^2$,那么 $S(72)=3+2=5$。小彼得,我希望你把这个数组对应的 $S$ 值求出来。

“小彼得,我希望你好好学习,要多看看一些数学书什么的!比如《初等数论》!在这周五
我会来检查你的作业情况的!”

但是在周五,小彼得收到了一台由 AMD 强力驱动的,并搭载了 Ubuntu 的计算机,根本无心
写那个即愚蠢又简单的、有手就行的问题。请问你能帮帮他吗?

%%%%%%%%%%%%%%%%%%%%%%%%%%%%%%%%%%%%%%%%%%%%%%%%%%%%%%%%%%%%%%%%%%%%%%%%%%%%%%%
\subsubsection{输入}
输入的第一行包含了一个整数 $t$($1\le t \le 10$),代表了有 $t$ 个测试样例,每一
个测试样例具有如下的格式:

每一个测试样例的第一行包含了两个整数,分别代表了$A_j$ 数组的数组长度 $L_j$($1
\le L_j \le 5 \times 10^{2}$,且 $1 \le t \times L_j \le 5 \times 10^{4}$),和
$R_j$(每一个测试样例的数字范围边界,$1 \le R_j \le 1 \times 10^{9}$)。

每一个测试样例的第二行包含 $L_j$ 个元素 $a_1,\, a_2,\, \ldots,\, a_{L_j}$,表示
$A_j$,对于任意 $i \in [1, L_j]$,有:$a_i \le R_j$。

%%%%%%%%%%%%%%%%%%%%%%%%%%%%%%%%%%%%%%%%%%%%%%%%%%%%%%%%%%%%%%%%%%%%%%%%%%%%%%%
\subsubsection{输出}
输出 $t$ 个结果,每个结果对应于输入的每一个测试样例。
对于每一个结果包含了 $L_j$ 个结果 $b_1,\, b_2,\, \ldots,\, b_{L_j}$。
其中 $b_i$ 为对应的 $a_i = p_1^{c_1}

%%%%%%%%%%%%%%%%%%%%%%%%%%%%%%%%%%%%%%%%%%%%%%%%%%%%%%%%%%%%%%%%%%%%%%%%%%%%%%%
\subsubsection{示例}
输入:
\begin{lstlisting}
1
10 100
3 12 4 34 25 78 90 45 23 21
\end{lstlisting}

输出:
\begin{lstlisting}
1 3 2 2 2 3 4 3 1 2
\end{lstlisting}

%%%%%%%%%%%%%%%%%%%%%%%%%%%%%%%%%%%%%%%%%%%%%%%%%%%%%%%%%%%%%%%%%%%%%%%%%%%%%%%
\subsubsection{说明}
对于 $3$,有:$3 = 3 \Longrightarrow S(3) = 1$。

对于 $12$,有:$12 = 2^2 \cdot 3 \Longrightarrow S(12) = 3$。

对于 $4$,有:$4 = 2^2 \Longrightarrow S(4) = 2$。

对于 $34$,有:$34 = 2 \cdot 17 \Longrightarrow S(34) = 2$。

对于 $25$,有:$25 = 5^2 \Longrightarrow S(25) = 2$。

对于 $78$,有:$78 = 2 \cdot 3^2 \cdot 13 \Longrightarrow S(78) = 3$。

对于 $90$,有:$90 = 2 \cdot 3^2 \cdot 5 \Longrightarrow S(90) = 4$。

对于 $45$,有:$45 = 3^2 \cdot 5 \Longrightarrow S(45) = 3$。

对于 $23$,有:$23 = 23 \Longrightarrow S(23) = 1$。

对于 $21$,有:$21 = 3 \cdot 7 \Longrightarrow S(21) = 2$。

