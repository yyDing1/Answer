\subsection{题面}

%%%%%%%%%%%%%%%%%%%%%%%%%%%%%%%%%%%%%%%%%%%%%%%%%%%%%%%%%%%%%%%%%%%%%%%%%%%%%%%
\subsubsection{Time Limit}
5s

%%%%%%%%%%%%%%%%%%%%%%%%%%%%%%%%%%%%%%%%%%%%%%%%%%%%%%%%%%%%%%%%%%%%%%%%%%%%%%%
\subsubsection{Memory Limit}
128M

%%%%%%%%%%%%%%%%%%%%%%%%%%%%%%%%%%%%%%%%%%%%%%%%%%%%%%%%%%%%%%%%%%%%%%%%%%%%%%%
\subsubsection{题目}
\begin{quote}
小心你的魔法把你给毁了。
\end{quote}

《炉石传说》是pzgg最喜欢玩的游戏,并且pzgg已经达到了炉火纯青的地步。前两天,炉石
的策划师给pzgg打电话,告知他要出一张新的魔法卡牌,想听听pzgg的意见。卡牌的作用
如下:

假设敌方战场上有 $n$ 只随从,第 $i$ 只随从($1 \leq i \leq n$)具有生命值 $a_i$。

当 $n \geq 3$ 时,己方英雄可以打出该卡牌,然后选择一个敌方随从 $i$($2 \leq i
\leq n-1$),可以直接将该随从秒杀(之后移出战场),并为己方英雄回复 $a_{i−1} \times
a_i \times a_{i+1}$ 的血量。

为了测试这张卡牌会不会违反游戏平衡,策划师给了pzgg若干个战场,并且给了pzgg无数
张该魔法卡牌,策划师想知道在每个战场上,己方英雄最多可回复多少滴血量。

pzgg是acm高手,觉得这太简单了,所以现在他想来考考聪明的你。

%%%%%%%%%%%%%%%%%%%%%%%%%%%%%%%%%%%%%%%%%%%%%%%%%%%%%%%%%%%%%%%%%%%%%%%%%%%%%%%
\subsubsection{输入}
输入包括 $2 \times T+1$ 行;
第一行包括一个整数$T$($1 \leq T \leq 10$),即以下有 $T$ 个战场需要你去求出结果;
接下去 $T$ 组数据,每组数据有两行,第一行是一个整数 $n$ (J$1 \leq n \leq 200$)
表示场上有 $n$ 个随从,第二行有 $n$ 个数,第 $i$ 个数 $a_i$($1 \leq a_i \leq 1
\times 10^5$)表示第 $i$ 个随从的生命值。

%%%%%%%%%%%%%%%%%%%%%%%%%%%%%%%%%%%%%%%%%%%%%%%%%%%%%%%%%%%%%%%%%%%%%%%%%%%%%%%
\subsubsection{输出}
对于每组数据输出一行,表示己方英雄可以最多回复的血量。

%%%%%%%%%%%%%%%%%%%%%%%%%%%%%%%%%%%%%%%%%%%%%%%%%%%%%%%%%%%%%%%%%%%%%%%%%%%%%%%
\subsubsection{示例}
输入:
\begin{lstlisting}
2
3
1 2 3
4
1 2 3 4
\end{lstlisting}

输出:
\begin{lstlisting}
6
32
\end{lstlisting}

将先手取的玩家称为先手玩家,后手取的玩家称为后手玩家

把问题归结于一下几种情况:

1. 若$n \% (m + 1) =0$,对于每次先手玩家删除文件的个数$x, \ 1 \leq x \leq m$,后手玩家只需删除个数为$m + 1 -x$即可,由$1 \leq x \leq m$,有$1 \leq m + 1 - x \leq m$,所以这种删法必定合法。因为每次先手玩家和后手玩家去完文件都少了$m + 1$个,故最后一次后手玩家一定能删完。后手必胜。
2. 若$n \% (m + 1) \neq 0$,对于先手玩家来说,他只需删$n \% (m + 1)$个文件,即可使文件总量满足情况1的条件,并且情况2的先手相当于先手删完一次之后情况1的后手,转化为情况1,所以先手必胜。

时间复杂度:$O(1)$
