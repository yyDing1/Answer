\subsection{题面}

%%%%%%%%%%%%%%%%%%%%%%%%%%%%%%%%%%%%%%%%%%%%%%%%%%%%%%%%%%%%%%%%%%%%%%%%%%%%%%%
\subsubsection{Time Limit}
5s

%%%%%%%%%%%%%%%%%%%%%%%%%%%%%%%%%%%%%%%%%%%%%%%%%%%%%%%%%%%%%%%%%%%%%%%%%%%%%%%
\subsubsection{Memory Limit}
128M

%%%%%%%%%%%%%%%%%%%%%%%%%%%%%%%%%%%%%%%%%%%%%%%%%%%%%%%%%%%%%%%%%%%%%%%%%%%%%%%
\subsubsection{题目}
\begin{quote}
小心你的魔法把你给毁了。
\end{quote}

《炉石传说》是pzgg最喜欢玩的游戏,并且pzgg已经达到了炉火纯青的地步。前两天,炉石
的策划师给pzgg打电话,告知他要出一张新的魔法卡牌,想听听pzgg的意见。卡牌的作用
如下:

假设敌方战场上有 $n$ 只随从,第 $i$ 只随从($1 \leq i \leq n$)具有生命值 $a_i$。

当 $n \geq 3$ 时,己方英雄可以打出该卡牌,然后选择一个敌方随从 $i$($2 \leq i
\leq n-1$),可以直接将该随从秒杀(之后移出战场),并为己方英雄回复 $a_{i−1} \times
a_i \times a_{i+1}$ 的血量。

为了测试这张卡牌会不会违反游戏平衡,策划师给了pzgg若干个战场,并且给了pzgg无数
张该魔法卡牌,策划师想知道在每个战场上,己方英雄最多可回复多少滴血量。

pzgg是acm高手,觉得这太简单了,所以现在他想来考考聪明的你。

%%%%%%%%%%%%%%%%%%%%%%%%%%%%%%%%%%%%%%%%%%%%%%%%%%%%%%%%%%%%%%%%%%%%%%%%%%%%%%%
\subsubsection{输入}
输入包括 $2 \times T+1$ 行;
第一行包括一个整数$T$($1 \leq T \leq 10$),即以下有 $T$ 个战场需要你去求出结果;
接下去 $T$ 组数据,每组数据有两行,第一行是一个整数 $n$ (J$1 \leq n \leq 200$)
表示场上有 $n$ 个随从,第二行有 $n$ 个数,第 $i$ 个数 $a_i$($1 \leq a_i \leq 1
\times 10^5$)表示第 $i$ 个随从的生命值。

%%%%%%%%%%%%%%%%%%%%%%%%%%%%%%%%%%%%%%%%%%%%%%%%%%%%%%%%%%%%%%%%%%%%%%%%%%%%%%%
\subsubsection{输出}
对于每组数据输出一行,表示己方英雄可以最多回复的血量。

%%%%%%%%%%%%%%%%%%%%%%%%%%%%%%%%%%%%%%%%%%%%%%%%%%%%%%%%%%%%%%%%%%%%%%%%%%%%%%%
\subsubsection{示例}
输入:
\begin{lstlisting}
2
3
1 2 3
4
1 2 3 4
\end{lstlisting}

输出:
\begin{lstlisting}
6
32
\end{lstlisting}

对于当前的值$val = k + 1$,将所有情况归纳与一下几种情况:

1. $val$是合数,所有比它小的数中必然存在他的因子$p, \ p < val, p > 2$,和一个素数。$gcd(p, val) = p > 1$,第一轮必然不会被感召,必定大于$1$轮。对于所有$gcd(val, x) > 1$,$gcd(val, val - 1) = 1$,第一轮$val - 1$一定会被感召,$gcd(val - 1, x) = 1$,所以所有数都会在第二轮被全部感召。答案为$2$
2. $val$是素数
   - 如果不存在$val$的倍数,对于所有$x$,$gcd(val, x)<val$为$val$的因子,因为$val$是素数,因子只有$1, val$所以$gcd(val, x) = 1$,一轮就可以传染全部。答案为$1$
   - 如果存在$val$的倍数,对于所有,$gcd(val, x)<val$为$val$的因子的情况同上,对于$gcd(val, x)=val$即$x$为$val$的倍数的情况,需要两次,证明同情况1​

判别素数:$O(\sqrt{n})$,判别倍数:$O(1)$

时间复杂度:$O(\sqrt{n})$
