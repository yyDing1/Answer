\subsection{题面}

%%%%%%%%%%%%%%%%%%%%%%%%%%%%%%%%%%%%%%%%%%%%%%%%%%%%%%%%%%%%%%%%%%%%%%%%%%%%%%%
\subsubsection{Time Limit}
1s

%%%%%%%%%%%%%%%%%%%%%%%%%%%%%%%%%%%%%%%%%%%%%%%%%%%%%%%%%%%%%%%%%%%%%%%%%%%%%%%
\subsubsection{Memory Limit}
128M

%%%%%%%%%%%%%%%%%%%%%%%%%%%%%%%%%%%%%%%%%%%%%%%%%%%%%%%%%%%%%%%%%%%%%%%%%%%%%%%
\subsubsection{题目}
小明决定走出自己的舒适圈,但是他很懒。所以,他决定让 wygg 和 sngg 帮忙督促他。

舒适区的范围可以假想为二维平面中以坐标轴原点为圆心,半径为 $d$ 的圆。刚开始,小
明的位置在 $(0,0)$ 位置(wygg和sngg轮流督促他,第一天是wygg)。每天wygg(sngg)
督促小明往 $x$ 轴的正方向或者 $y$ 轴的正方向移动 $k$ 个单位距离。(如果小明距离
原点的距离超过了 $d$,则代表小明走出了自己的舒适圈,游戏结束)。

由于小明会暗中记恨那个让他走出舒适区的人,wygg 和 sngg 都不想被小明记恨,所以他
们都不想走最后一步。wygg 和 sngg 都是绝顶聪明的人,保证他们每一步都做出最优的选
择,看看谁会走最后一步,即让小明走出舒适圈的人是谁。

%%%%%%%%%%%%%%%%%%%%%%%%%%%%%%%%%%%%%%%%%%%%%%%%%%%%%%%%%%%%%%%%%%%%%%%%%%%%%%%
\subsubsection{输入}
第一行包括一个数字 $t$(代表测试数据的数量)。

每一行包括两个被空格分开的数字 $d$($1 \leq d \leq 100000$),$k$($1 \leq k \leq
d$)。

%%%%%%%%%%%%%%%%%%%%%%%%%%%%%%%%%%%%%%%%%%%%%%%%%%%%%%%%%%%%%%%%%%%%%%%%%%%%%%%
\subsubsection{输出}
对于每一组数据,如果 wygg 走最后一步输出 \verb|"wygg"|,否则则输出 \verb|"sngg"|。

%%%%%%%%%%%%%%%%%%%%%%%%%%%%%%%%%%%%%%%%%%%%%%%%%%%%%%%%%%%%%%%%%%%%%%%%%%%%%%%
\subsubsection{示例}
输入:
\begin{lstlisting}
3
2 1
5 2
10 3
\end{lstlisting}

输出:
\begin{lstlisting}
wygg
sngg
wygg
\end{lstlisting}

%%%%%%%%%%%%%%%%%%%%%%%%%%%%%%%%%%%%%%%%%%%%%%%%%%%%%%%%%%%%%%%%%%%%%%%%%%%%%%%
\subsection{题解}
这是一个博弈论的题目。我们可以给每一个点打上\underline{必输}或者\underline{必赢}
状态。可以列矩阵如下:

\def\rddots#1{\cdot^{\cdot^{\cdot^{#1}}}}
$$\begin{bmatrix}
           &        & \rdots  & Y      & X       \\
           &        & \cdots  & X      & Y       \\
    \vdots & \vdots & \rddots & \vdots & \vdots  \\
    Y      & X      & \cdots                     \\
    X      & Y      & \cdots                     \\
\end{bmatrix}$$

博弈论的状态图是一个有向无环图,且又性质为:
\begin{itemize}
    \item 没有后继状态的状态是必败状态。
    \item 一个状态是必胜状态,当且仅当至少存在一个必败状态为它的后继状态。
    \item 一个状态是必败状态,当且仅当它的所有后继状态均为必胜状态。
\end{itemize}

其中标记 $X$ 的都是 wygg 面临的初始状态。如果最后一个 $X$ 之后没有后继状态,那么
所有的 $X$ 标记上的状态就是必输状态。否则它能有向地到达 $Y$ 状态,而因为该 $X$ 状
态是最后一个 $X$ 状态,所以它无法到达下一个坐标为 $(x+1,\ x+1)$ 的点,又通过简单的
几何学,$(x,\ x+1)$ 也不能到达 $(x,\ x+2)$,同理 $(x+1,\ x)$ 也没有后继状态,为必
输状态。

不用考虑其他状态,这个带子上所有的状态点即为判断起始点为是否为必赢状态/必输状态
的充分条件。我们可以断言:
\begin{itemize}
    \item 如果最后一个 $X$ 点无后继节点,则所有的 $X$ 节点,包含起始点均为必败节
        点。
    \item 如果最后一个 $X$ 点有后继节点 $Y$,且根据几何知识能证明该 $Y$ 节点无后
        继节点,即该 $Y$ 节点为必输节点。又因为 $X$ 为的所有后继节点为必输节点,
        所以说该 $X$ 节点为必赢节点,且所有 $X$ 都为必赢节点。
\end{itemize}

\subsection{参考代码}
如下,其中 $t$ 满足 $|({tk},\ {tk})| \leq d$。而 \verb|more| 表示最后的 $X$ 节点
之后是否有下一个 $Y$ 节点。
\begin{lstlisting}
using ll=long long;

void swaper() {
    ll t;
    for(scanf("%lld",&t);t--;){
        ll d,k;
        scanf("%lld %lld",&d,&k);

        ll t=(ll)floor(d*1.0/sqrt(2)/k)-1;
        while((t+1)*k*(t+1)*k+(t+1)*k*(t+1)*k<=d*d)t++;

        bool more=(t*k*t*k+(t+1)*k*(t+1)*k<=d*d);

        printf((!more)?"wygg\n":"sngg\n");
    }
}
\end{lstlisting}

